%% Preset
\documentclass[a4paper, twoside]{report}

\usepackage[red]{unifront}
\usepackage{hyperref}

%% Document information
\author{Dev M.}
\date{07/08/2023}

%% Document start from here
\begin{document}

\newtitle{CN-Lecture 1}{5}{\currentchapter{Course struct}{Intro to Network}}

%% Chapter1 start from here
\newchapter{Course struct}{1}{}

%% section1 start from here
\newsection{}{5}{\currentsubsections{Grading}{Schedule}{Textbook}}
%% Class Quiz -> 10 \% (n-1, 3)
% Midsem -> 15 \%
% Endsem -> 35 \%
% Assignments -> 40 \% (n, 4)
\newsubsection{Grading}
\begin{itemize}
    \item 10 \% $\rightarrow$ Quizzes (n-1, 3)
    \item 15 \% $\rightarrow$ Midsem
    \item 35 \% $\rightarrow$ Endsem
    \item 40 \% $\rightarrow$ Assignments (n, 4)
\end{itemize}


\newsubsection{Schedule}
Monday to Wednesday $\rightarrow$ 9:00 to 11 (lect)

Prof Office hours  $\rightarrow$ 12:00 to 13:00 (mon)

Head TA $\rightarrow$ Gopi

\newsubsection{Textbook}
\begin{itemize}
    \item J.F. Kurose and K.W. Ross, Computer Networking: A Top-Down Approach, 6th Edition, Pearson, 2012.
    \item \url{https://gaia.cs.umass.edu/kurose_ross/ppt.htm}
    \item L L Peterson and B S Davie, Computer Networks: A Systems Approach, 5th Edition, Morgan Kaufmann, 2011.
    \item  A. S. Tanenbaum and D. J. Wetherall, Computer Networks, 5th Edition, Pearson, 2011.
\end{itemize}


%% Chapter2 start from here
\newchapter{Intro to Network}{9}{\currentsections{Common Terminology}{High Level View}}

%% Common Terminology start from here
\newsection{Common Terminology}{5}{\currentsubsections{Router}{IP Address}{Port}{MAC Address}{DNS}{NAT}{Connection}}

%% Router subsection start from here
\newsubsection{Router}
\begin{itemize}
    \item A device that connects two or more networks.
    \item A device that forwards data packets between computer networks.
    \item Helps to determine the best path for the data packets to reach the destination.
\end{itemize}

%% IP Address subsection start from here
\newsubsection{IP Address}
\begin{itemize}
    \item A unique number assigned to each host.
    \item Generally written in dotted decimal notation.
    \item Helps in specifying each unique user on the network/internet.
    \item Global Identifier for a device.
    \item In IPv4, 32 bits are divided into 4 octets. \\
            Example: 127.123.123.123
    \item In IPv6, 128 bits are divided into 8 octets.
    \item There are two types of IP addresses:
        \begin{itemize}
            \item Public IP Address: Assigned by ISP (Internet Service Provider) and visible to each device connected to the internet.
            \item Private IP Address: Assigned by the router and visible only to the devices connected to the router.
        \end{itemize}
        For proof you can match your local IP Address (Using Ipconfig) with the IP Address shown on \url{https://whatismyipaddress.com/}
\end{itemize}

%% Port subsection start from here
\newsubsection{Port}
\begin{itemize}
    \item A 16-bit number assigned to each process.
    \item Port has nothing to do with process ID (PID), it is just a number assigned to each process using the network.
    \item Helps in specifying each unique process on the network/internet.
    \item No two processes can have the same port number.
    \item There are two types of ports:
        \begin{itemize}
            \item Well-known ports: 0 to 1023
            \item Registered ports/User-Defined ports: 1024 to 49151
            \item Dynamic ports: 49152 to 65535
        \end{itemize}
    \item Well known ports are assigned to processes by the IANA (Internet Assigned Numbers Authority).
    \item These ports are used by system processes that provide widely used types of network services.
    \item Examples are: FTP (21), SSH (22), Telnet (23), SMTP (25), HTTP (80), HTTPS (443), etc.
    \item Rest of the ports are used by user-defined processes like Chrome, Firefox, etc.
\end{itemize}

%% MAC Address subsection start from here
\newsubsection{MAC Address}
\begin{itemize}
    \item A unique number assigned to each network interface.
    \item Local Identifier for a device.
    \item Generally 48 bits long. Written in hexadecimal notation.
\end{itemize}

%% DNS subsection start from here
\newsubsection{DNS}
\begin{itemize}
    \item Domain Name System.
    \item A system that maps and fetches domain names/URLs to IP Addresses.
    \item DNS IP is preconfigured/hardcoded in the router and is fetched from the ISP\@.
\end{itemize}

%% NAT subsection start from here
\newsubsection{NAT}
\begin{itemize}
    \item Network Address Translation.
    \item A system that maps and fetches private IP Addresses to public IP Addresses.
    \item NAT IP is preconfigured/hardcoded in the router and is fetched from the ISP\@.
\end{itemize}

%% Connections subsection start from here
\newsubsection{Connection}
\begin{itemize}
    \item Connection is a 4-tuple of (source IP, source port, destination IP, destination port).
    \item Connection is uniquely identified by the 4-tuple and is created upon a successful handshake between the client and the server.
\end{itemize}

%% High Level View section start from here
\newsection{High Level View}{5}{\currentsubsections{Connect to a website}}

%% Connect to a website subsection start from here
\newsubsection{Connect to a website}
\begin{itemize}
    \item The process:
    \begin{itemize}
        \item User types the URL in the browser.
        \item Browser fetches the IP Address of the URL from the DNS\@.
        \item Browser sends a request  o the IP Address on port 80 (HTTP) or 443 (HTTPS).
        \item Request is sent to your ISP's router.
        \item Router maps the private IP Address to a public IP Address using NAT\@.
        \item Router forwards the request to the public IP Address.
        \item Server responds to the request.
        \item Router maps the public IP Address to a private IP Address using NAT\@.
        \item Router forwards the response to the private IP Address.
        \item Use the specifed port, if any, to forward the request and response.
        \item Browser renders the response.
    \end{itemize}
    \item This process is repeated for each request and response and is called a connection.
    \item The Endpoints are:
    \begin{itemize}
        \item Client's Global IP Address and Port.
        \item Server's Global IP Address and HTTP/HTTPS/Well-known port.
    \end{itemize}
    Both these Endpoints are known as socket addresses.
\end{itemize}

\end{document}
